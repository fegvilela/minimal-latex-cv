%----------------------------------------------------------------------------------------
%	DOCUMENT DEFINITION
%----------------------------------------------------------------------------------------

% we use article class because we want to fully customize the page
\documentclass[10pt,A4]{article}


%----------------------------------------------------------------------------------------
%	ENCODING
%----------------------------------------------------------------------------------------

%we use utf8 since we want to build from any machine
\usepackage[utf8]{inputenc}

%----------------------------------------------------------------------------------------
%	LOGIC
%----------------------------------------------------------------------------------------

\usepackage{xifthen}
\usepackage{calc}
\usepackage[hidelinks]{hyperref}

%----------------------------------------------------------------------------------------
%	FONT
%----------------------------------------------------------------------------------------

% some tex-live fonts - choose your own

%\usepackage[defaultsans]{droidsans}
%\usepackage[default]{comfortaa}
%\usepackage{cmbright}
\usepackage[default]{raleway}
%\usepackage{fetamont}
%\usepackage[default]{gillius}
%\usepackage[light,math]{iwona}
%\usepackage[thin]{roboto}

% set font default
\renewcommand*\familydefault{\sfdefault}
\usepackage[T1]{fontenc}

% more font size definitions
\usepackage{moresize}

% font icons package
\usepackage{fontawesome}

%----------------------------------------------------------------------------------------
%	PAGE LAYOUT  DEFINITIONS
%----------------------------------------------------------------------------------------

%define page styles using geometry
\usepackage[a4paper]{geometry}

% for example, change the margins to 2 inches all round
\geometry{top=2cm, bottom=2cm, left=2cm, right=2cm}

% use customized header
\usepackage{fancyhdr}
\pagestyle{fancy}

%less space between header and content
\setlength{\headheight}{-5pt}

% customize header entries
\lhead{}
\rhead{}
\chead{}

%indentation is zero
\setlength{\parindent}{0mm}

%----------------------------------------------------------------------------------------
%	GRAPHICS DEFINITIONS
%----------------------------------------------------------------------------------------

% for drawing graphics and charts
\usepackage{tikz}
\usetikzlibrary{shapes, backgrounds}

% use to vertically center content
% credits to: http://tex.stackexchange.com/questions/7219/how-to-vertically-center-two-images-next-to-each-other
\newcommand{\vcenteredinclude}[1]{\begingroup
	\setbox0=\hbox{\includegraphics{#1}}%
	\parbox{\wd0}{\box0}\endgroup}

% use to vertically center content
% credits to: http://tex.stackexchange.com/questions/7219/how-to-vertically-center-two-images-next-to-each-other
\newcommand*{\vcenteredhbox}[1]{\begingroup
	\setbox0=\hbox{#1}\parbox{\wd0}{\box0}\endgroup}

%----------------------------------------------------------------------------------------
%	ICON-SET EMBEDDING
%----------------------------------------------------------------------------------------

% at this point we simplify our icon-embedding by simply referring to a set of png images.
% if you find a good way of including svg without conflicting with other packages you can
% replace this part
\newcommand{\icon}[2]{\colorbox{black}{\makebox(#2, #2){\textcolor{white}{\large\csname fa#1\endcsname}}}}	%icon shortcut
\newcommand{\icontext}[3]{ 						%icon with text shortcut
	\vcenteredhbox{\icon{#1}{#2}}\hspace{0.2cm}\vcenteredhbox{\textcolor{black}{#3}}
}

%----------------------------------------------------------------------------------------
% 	HEADER
%----------------------------------------------------------------------------------------

% remove top header line
\renewcommand{\headrulewidth}{0pt}

%remove botttom header line
\renewcommand{\footrulewidth}{0pt}

%remove pagenum
\renewcommand{\thepage}{}

%remove section num
\renewcommand{\thesection}{}


%----------------------------------------------------------------------------------------
%
% 	TIKZ GRAPHICS
%
%----------------------------------------------------------------------------------------

\newcounter{a}
\newcounter{b}
\newcounter{c}
\newcounter{barcount}

%----------------------------------------------------------------------------------------
% 	BAR CHART
%----------------------------------------------------------------------------------------

% draw a bar chart
% param 1: width
% param 2: height
% param 3: border color
% param 4: label text color
% param 5: label bg color
% param 6: cat 1 color
\newenvironment{barchart}[8]{

	\newcommand{\barwidth}{0.35}
	\newcommand{\barsep}{0.2}

	% param 1: overall percent
	% param 2: label
	% param 3: cat 1 percent
	% param 4: cat 2 percent
	% param 5: cat 3 percent
	\newcommand{\baritem}[5]{

		\pgfmathparse{##3+##4+##5}
		\let\perc\pgfmathresult

		\pgfmathparse{#2}
		\let\barsize\pgfmathresult

		\pgfmathparse{\barsize*##3/100}
		\let\barone\pgfmathresult

		\pgfmathparse{\barsize*##4/100}
		\let\bartwo\pgfmathresult

		\pgfmathparse{\barsize*##5/100}
		\let\barthree\pgfmathresult

		\pgfmathparse{(\barwidth*\thebarcount)+(\barsep*\thebarcount)}
		\let\barx\pgfmathresult

		\filldraw[fill=#6, draw=none] (0,-\barx) rectangle (\barone,-\barx-\barwidth);
		\filldraw[fill=#7, draw=none] (\barone, -\barx) rectangle (\barone+\bartwo,-\barx-\barwidth);
		\filldraw[fill=#8, draw=none] (\barone+\bartwo,-\barx ) rectangle (\barone+\bartwo+\barthree,-\barx-\barwidth);

		\node [label=180:\colorbox{#5}{\textcolor{#4}{##2}}] at (0,-\barx-0.175) {};
		\addtocounter{barcount}{1}
	}
	\begin{tikzpicture}
	\setcounter{barcount}{0}

}
{\end{tikzpicture}}

%----------------------------------------------------------------------------------------
% 	BUBBLE CHART
%----------------------------------------------------------------------------------------
\newcommand{\bubble}[5]{
	\definecolor{tmpcol}{RGB}{50,50,#5}
	\definecolor{greenbox}{RGB}{136, 160, 150}
	% slice
	\filldraw[fill=black,draw=none] (#1,0.5) circle (#3);

	% outer label
	\node[label=\textcolor{black}{#4}] at (#1,0.7) {};
}

\newcommand{\bubbles}[2]{
	%reset counters
	\setcounter{a}{0}
	\setcounter{c}{150}
	\begin{tikzpicture}[scale=3]
	\foreach \p/\t in {#1} {
		\addtocounter{a}{1}
		\bubble{\thea/2}{\theb}{\p/25}{\t}{1\p0}
	}
	\end{tikzpicture}
}


%----------------------------------------------------------------------------------------
%	custom sections
%----------------------------------------------------------------------------------------

% create a coloured box with arrow and title as cv section headline
% param 1: section title
%
\newcommand{\cvsection}[1] {
	\textcolor{white}{\MakeUppercase{\textbf{#1}}}
}

\newcommand{\cvsect}[1]{
	\colorbox{black}{{\cvsection{#1}}}\\\\%
}

%----------------------------------------------------------------------------------------
% CUSTOM LOREM IPSUM
%----------------------------------------------------------------------------------------
\newcommand{\lorem}{Lorem ipsum dolor sit amet, consectetur adipiscing elit. Donec a diam lectus.}

%----------------------------------------------------------------------------------------
% ENTRY LIST
%----------------------------------------------------------------------------------------
\usepackage{tabularx}

\setlength{\tabcolsep}{0pt}
\newenvironment{entrylist}{%
	\begin{tabular*}{\textwidth}[t]{@{\extracolsep{\fill}}ll}
	}{%
	\end{tabular*}
}

\newcommand{\entry}[4]{%
	\parbox[t]{3.5cm}{%
		#1%
	}%
	&\parbox[t]{14cm}{%
		\textbf{#2}%
		\hfill%
		{\footnotesize \textbf{\textcolor{black}{#3}}}\\%
		#4%
	}\\\\}

\newcommand{\slashsep}{
	\hspace{2mm}/\hspace{2mm}
}

%----------------------------------------------------------------------------------------
%	DOCUMENT CONTENT
%----------------------------------------------------------------------------------------
\begin{document}

	%----------------------------------------------------------------------------------------
	%	TITLE HEADLINE
	%----------------------------------------------------------------------------------------
	\begin{minipage}[t]{0.61\textwidth}\hrule height 0pt width 0pt%
		\colorbox{black}{{\HUGE\textcolor{white}{\textbf{\MakeUppercase{Fernanda \break Vilela}}}}}%

		\vspace{1mm}\LARGE{Dev/Gestora de Projetos}
	\end{minipage}%
	\begin{minipage}[t]{0.2\textwidth}\hrule height 0pt width 0pt%
		\small%
		\icontext{MapMarker}{12}{Brasília, DF}\\
		\icontext{Phone}{12}{(61)98410-5469}\\
		\icontext{Heartbeat}{12}{27 anos}\\
	\end{minipage}%
	\begin{minipage}[t]{0.3\textwidth}\hrule height 0pt width 0pt%
		\small%
		\icontext{At}{12}{\href{mailto:fegvilela@gmail.com}{fegvilela@gmail.com}}\\
		\icontext{Github}{12}{\href{https://github.com/fegvilela}{github.com/fegvilela}}\\
		\icontext{Linkedin}{12}{\href{https://www.linkedin.com/in/fernanda-vilela-1bba6b8b/}{fernanda-vilela-1bba6b8b}}\\
	\end{minipage}%

	% manage space by reducing font size
	\small%
	\vspace{0.1cm}

	%----------------------------------------------------------------------------------------
	%	SKILLS AND TECHNOLOGIES
	%----------------------------------------------------------------------------------------

	\cvsect{Quem sou eu?}%
	\begin{minipage}[t]{1.03\textwidth}%
		Engenheira eletricista com experiência em gestão e desenvolvimento de software e hardware. Tenho interesse em ambientes e em projetos que promovem a liberdade criativa e a inovação. Busco projetos em que possa trabalhar com desenvolvimento e/ou gestão de projetos. Tenho forte interesse na área de ML e busco entender o cenário no qual um projeto se encontra, gosto de atuar no planejamento, na conexão com as demais áreas da empresa e no posicionamento de mercado.\\
	\end{minipage}%
	\hfill
	\cvsect{Experiência}
	\begin{entrylist}
		\entry
		{set/20 – atualmente\\}
		{Eng. de Inteligência/Bolsista de P\&D CNPq}
		{\href{https://next-cam.com}{NEXTCAM - Visão Computacional e IA aplicada}}
		{Meu trabalho é focado na área de desenvolvimento de inteligência, projetando e implementando soluções e ferramentas que otimizem o processo de MLOps do produto. O produto aplica visão computacional para garantir que normas e processos de segurança em indústrias. \\
			\texttt{Node JS} \slashsep\texttt{Postgres}\slashsep\texttt{TensorFlow}\slashsep\texttt{GCP}\slashsep\texttt{Bash}}
		\entry
		{abr/20 – atualmente\\\footnotesize{ago/18 – nov/18}}
		{Desenvolvedora Front-End e Gestora de Projetos}
		{\href{https://humanoide.co/}{HUMANOIDE - Estúdio de Criação Digital}}
		{Tenho buscado trazer reestruturação na parte de gestão de desenvolvimento e de produto, mas minha tarefa principal é desenvolvimento de interfaces web e mobile em alguns projetos da empresa, colaborando também na área de usabilidade.\\
			\texttt{HTML(Slim) SASS} \slashsep\texttt{Figma}\slashsep\texttt{React Native}\slashsep\texttt{JS}\slashsep\texttt{Ruby on Rails}\slashsep\texttt{HubSpot}}
		\entry
		{jan/20 – mai/20\\\footnotesize{dez/17 - abr/18}}
		{Engenheira de Desenvolvimento}
		{\href{https://gembrap.com/}{GEMBRAP S.A. - Geradores de energia e motores}}
		{Desenvolvimento de motores elétricos de baixo consumo para ventiladores, compressores e exaustores, atuando principalmente na área de eletrônica de potência.\\
			\texttt{Eletrônica de potência}\slashsep\texttt{Prototipagem}\slashsep\texttt{SCRUM}\slashsep\texttt{Impressão 3D}}
		\entry
		{dez/16 – jul/19\\}
		{Co-fundadora, desenvolvedora e gestora de projetos}
		{\href{https://www.instagram.com/roadiebot/?hl=pt-br}{ROADIEBOT - Tecnologia para música}}
		{Participei de todo o processo de validação e construção da empresa, além de participar da gestão de desenvolvimento, desenvolvimento full-stack do app desktop e do hardware de automação.\\
			\texttt{node.js}\slashsep\texttt{ElectronJS}\slashsep\texttt{HTML CSS JQuery}\slashsep\texttt{Python}\slashsep\texttt{Arduino}\slashsep\texttt{Impressão 3D}\slashsep\texttt{Empreendedorismo científico}}
		\entry
		{nov/18 – mar/19\\}
		{Desenvolvedora Java Jr.}
		{\href{https://www.techis.inf.br/}{TECHIS Intelligent Solutions - Monitoramento de Transporte}}
		{Trabelhei na concepção, planejamento e homologação do novo produto da empresa, operando na integração entre o desenvolvimento de hardware (localizador de caminhão) e de software (sistema de monitoramento e gestão), em conjunto com empresa parceira.\\
			\texttt{Java}\slashsep\texttt{Arduino}\slashsep\texttt{Design Sprint}}
	\end{entrylist}

	\cvsect{Educação}
	\begin{entrylist}
		\entry
		{2011 – 2017}
		{Engenharia Elétrica, Graduação}
		{Universidade de Brasília}
		{Trabalho de Conclusão de Curso: Sistema de reconhecimento de padrões em tipografia para ensino a deficientes visuais. Orientadora: Prof. Dr. Mylène Christine Farias \\ \texttt{OpenCV}\slashsep\texttt{Scikit-learn}\slashsep\texttt{Python}}
		\entry
		{2014 - 2015}
		{Ciência sem Fronteiras UK, Graduação Sanduíche}
		{Brunel University London}
		{Trabalho de Conclusão de Curso: Simulação para sistema de filas para utilização de codec H264. Orientador: Prof. Dr. John Cosmas \\\texttt{Matlab}\slashsep\texttt{Processamento de vídeo}}
		\end{entrylist}

	\begin{minipage}[t]{0.7\textwidth}\hrule height 0pt width 0pt%
		\cvsect{Projetos}
		- Ensino de programação para adolescentes \texttt{(Python/Django)}\\
		- Professora particular de matemática e física por 14 anos\\
		- Simulação de criação de imagens panorâmicas \texttt{(Matlab)}\\
		- Desenvolvimento de plataforma de RV para ensino de motores elétricos \texttt{(Unity 3D, Blender, C\#, JS)}\\
		- Tradutora voluntária (inglês) no blog {\href{https://reforma21.org/}{reforma21.org}}
	\end{minipage}%
	\hspace{1cm}
	\begin{minipage}[t]{0.27\textwidth}\hrule height 0pt width 0pt%
		\cvsect{Idiomas}
		\textbf{Inglês} - fluente\\
		\textbf{Italiano} - intermediário\\\\
		
	\end{minipage}%

\end{document}